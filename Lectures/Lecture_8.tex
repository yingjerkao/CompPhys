% $Header: /home/vedranm/bitbucket/beamer/solutions/generic-talks/generic-ornate-15min-45min.en.tex,v 90e850259b8b 2007/01/28 20:48:30 tantau $
%\documentclass{beamer}
\documentclass[handout]{beamer}

% This file is a solution template for:

% - Giving a talk on some subject.
% - The talk is between 15min and 45min long.
% - Style is ornate.



% Copyright 2004 by Till Tantau <tantau@users.sourceforge.net>.
%
% In principle, this file can be redistributed and/or modified under
% the terms of the GNU Public License, version 2.
%
% However, this file is supposed to be a template to be modified
% for your own needs. For this reason, if you use this file as a
% template and not specifically distribute it as part of a another
% package/program, I grant the extra permission to freely copy and
% modify this file as you see fit and even to delete this copyright
% notice. 

\mode<presentation>
{
  \usetheme{Warsaw}
  % or ...

  \setbeamercovered{transparent}
  % or whatever (possibly just delete it)
}
\setbeamertemplate{navigation symbols}{} 

\usepackage[english]{babel}
% or whatever

\usepackage[latin1]{inputenc}
% or whatever
\useoutertheme{default}

\usepackage{times}
\usepackage[T1]{fontenc}
% Or whatever. Note that the encoding and the font should match. If T1
% does not look nice, try deleting the line with the fontenc.
\newcommand{\beforeverb}{\footnotesize}
\newcommand{\afterverb}{\normalsize}

\title[Classes and Objects: Data Encapsulation and Methods] % (optional, use only with long paper titles)
{Lecture 8}

\subtitle
{Classes and Objects} % (optional)

\author[Ying-Jer Kao] % (optional, use only with lots of authors)
{Ying-Jer Kao}
% - Use the \inst{?} command only if the authors have different
%   affiliation.

\institute[National Taiwan University] % (optional, but mostly needed)
{
  Department of Physics\\
 National Taiwan University
  }
% - Use the \inst command only if there are several affiliations.
% - Keep it simple, no one is interested in your street address.

\date[Numerical Analysis and Programming] % (optional)
{\today}

\subject{Talks}
% This is only inserted into the PDF information catalog. Can be left
% out. 



% If you have a file called "university-logo-filename.xxx", where xxx
% is a graphic format that can be processed by latex or pdflatex,
% resp., then you can add a logo as follows:

% \pgfdeclareimage[height=0.5cm]{university-logo}{university-logo-filename}
% \logo{\pgfuseimage{university-logo}}



% Delete this, if you do not want the table of contents to pop up at
% the beginning of each subsection:
%\AtBeginSubsection[]
%{
%  \begin{frame}<beamer>{Outline}
%    \tableofcontents[currentsection,currentsubsection]
%  \end{frame}
%}


% If you wish to uncover everything in a step-wise fashion, uncomment
% the following command: 

%\beamerdefaultoverlayspecification{<+->}


\begin{document}

\begin{frame}
  \titlepage
\end{frame}

\begin{frame}{Outline}
  \tableofcontents
  % You might wish to add the option [pausesections]
\end{frame}


% Since this a solution template for a generic talk, very little can
% be said about how it should be structured. However, the talk length
% of between 15min and 45min and the theme suggest that you stick to
% the following rules:  

% - Exactly two or three sections (other than the summary).
% - At *most* three subsections per section.
% - Talk about 30s to 2min per frame. So there should be between about
%   15 and 30 frames, all told.

\section[Classes]{Classes}
\subsection[Classes]{Classes}
\begin{frame}[fragile]
\frametitle{Python and OOP}

Python is an \alert{object-oriented programming language}. 
\begin{block}{OOP}
\begin{itemize}

\item Programs are made up of \alert{object definitions and function
definitions}, and most of the computation is expressed in terms
of operations on objects.

\item Each object definition corresponds to some \alert{object} or  \alert{concept}
in the real world, and the functions that operate on that object
correspond to the ways real-world objects \alert{interact}.

\end{itemize}

\end{block}

\end{frame}
\begin{frame}[fragile]
\frametitle{Class}
\begin{itemize}

\item We want to create new data types which group together information such that it makes sense.
\item  A user-defined type is also called a 
\alert{class}.
\begin{block}{Point Class}
\beforeverb
\begin{verbatim}
class Point:
    """represents a point in 2-D space"""
    pass
\end{verbatim}
\afterverb
\begin{itemize}
\item We defined a new class {\tt Point},
which is an {\tt object}, which is a \alert{built-in type}.
\item Later we can add functions and variables into the {\tt Point} class.
\end{itemize}
\end{block}


\end{itemize}

\end{frame}
\begin{frame}[fragile]
\frametitle{Instance}
\begin{itemize}
\item To create a new object of class {\tt Point}, use it just like a function.
\beforeverb
\begin{verbatim}
>>> blank = Point()
>>> print blank
<__main__.Point instance at 0xb7e9d3ac>
\end{verbatim}
\afterverb
 
\item Creating a new object is called
\alert{instantiation}, and the object is an \alert{ instance} of
the class.

\end{itemize}
\end{frame}
\subsection[Attributes]{Attributes}


\begin{frame}[fragile]
\frametitle{Attributes}
\begin{itemize}
\item Values of an instance can be assigned using dot notation.

\beforeverb
\begin{verbatim}
>>> blank.x = 3.0
>>> blank.y = 4.0
\end{verbatim}
\afterverb

\item The value of an attribute can be accessed using the same syntax:
\begin{block}{}
\beforeverb
\begin{verbatim}
>>> print blank.y
4.0
>>> x = blank.x
>>> print x
3.0
>>> print '(%g, %g)' % (blank.x, blank.y)
(3.0, 4.0)
>>> distance = math.sqrt(blank.x**2 + blank.y**2)
>>> print distance
5.0
\end{verbatim}
\afterverb
\end{block}
%
\end{itemize}
\end{frame}
\begin{frame}[fragile]
\frametitle{Copying}
\begin{itemize}
\item Aliasing can make a program difficult to read because changes
in one place might have unexpected effects in another place.

\item Copying an object is often an alternative to aliasing.
\item The {\tt copy} module contains a function called {\tt copy} that
can duplicate any object.
\begin{block}{{\tt copy} module}
\beforeverb
\begin{verbatim}
>>> p1 = Point()
>>> p1.x = 3.0
>>> p1.y = 4.0
>>> import copy
>>> p2 = copy.copy(p1)
\end{verbatim}
\afterverb
%
 {\tt p1} and {\tt p2} contain the \alert{same data}, but they are
not the same {\tt Point}.
\end{block}
\end{itemize}
\end{frame}
\begin{frame}[fragile]
\frametitle{ Equality}
\begin{block}{}
\beforeverb
\begin{verbatim}
>>> print_point(p1)
(3.0, 4.0)
>>> print_point(p2)
(3.0, 4.0)
>>> p1 is p2
False
>>> p1 == p2
False
\end{verbatim}
\afterverb
\end{block}
%
\begin{itemize}
\item 
The {\tt is} operator indicates that {\tt p1} and {\tt p2} are not the
same object, which is what we expected.  But you might have expected
{\tt ==} to yield {\tt True} because these points contain the same
data. 
 \item The default behavior of the {\tt ==} operator is the same
as the {\tt is} operator; it checks object identity, not object
equivalence.  This behavior can be changed. 
\end{itemize}
\end{frame}

\begin{frame}[fragile]
\frametitle{Methods}
\begin{itemize}
\item Methods are semantically the same as functions, but there are
two syntactic differences:

\begin{itemize}

\item Methods are defined \alert{inside} a class definition in order
to make the relationship between the class and the method explicit.

\item The syntax for invoking a method is \alert{different} from the
syntax for calling a function.

\end{itemize}
\end{itemize}
\end{frame}

\begin{frame}[fragile]
\frametitle{Special Methods}
\begin{itemize}
\item \verb" __init__": Class instantiation automatically invokes \verb!__init__()! for the newly-created class instance.
\item \verb"__str__":  Returns a string representation of an object.
\item \verb"__cmp__": Redefines comparison operations.	
\item \verb"__eq__": Redefines equality.
\item \verb"__add__": Redefines addition (concatenation).
\item \verb"__mul__": Redefines multiplication (repetition).
\item Others...

\end{itemize}
\end{frame}

\begin{frame}[fragile]
\frametitle{Inheritance}
\end{frame}

\end{document}


