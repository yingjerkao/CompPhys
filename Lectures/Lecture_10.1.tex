% $Header: /home/vedranm/bitbucket/beamer/solutions/generic-talks/generic-ornate-15min-45min.en.tex,v 90e850259b8b 2007/01/28 20:48:30 tantau $
\documentclass{beamer}
%\documentclass[handout]{beamer}
\usefonttheme[onlymath]{serif}
% This file is a solution template for:
\usepackage{algorithm}
\usepackage{algpseudocode}
% - Giving a talk on some subject.
% - The talk is between 15min and 45min long.
% - Style is ornate.



% Copyright 2004 by Till Tantau <tantau@users.sourceforge.net>.
%
% In principle, this file can be redistributed and/or modified under
% the terms of the GNU Public License, version 2.
%
% However, this file is supposed to be a template to be modified
% for your own needs. For this reason, if you use this file as a
% template and not specifically distribute it as part of a another
% package/program, I grant the extra permission to freely copy and
% modify this file as you see fit and even to delete this copyright
% notice. 

\mode<presentation>
{
  \usetheme{Warsaw}
  % or ...

  \setbeamercovered{transparent}
  % or whatever (possibly just delete it)
}
\setbeamertemplate{navigation symbols}{} 

\usepackage[english]{babel}
% or whatever

\usepackage[latin1]{inputenc}
% or whatever
\useoutertheme{default}

\usepackage{times}
\usepackage[T1]{fontenc}
% Or whatever. Note that the encoding and the font should match. If T1
% does not look nice, try deleting the line with the fontenc.
\newcommand{\beforeverb}{\footnotesize}
\newcommand{\afterverb}{\normalsize}

\title[Crash Course on Numpy ] % (optional, use only with long paper titles)
{Lecture 10.1}

\subtitle
{Crash Course on Numpy} % (optional)

\author[Ying-Jer Kao] % (optional, use only with lots of authors)
{Ying-Jer Kao}
% - Use the \inst{?} command only if the authors have different
%   affiliation.

\institute[National Taiwan University] % (optional, but mostly needed)
{
  Department of Physics\\
 National Taiwan University
  }
% - Use the \inst command only if there are several affiliations.
% - Keep it simple, no one is interested in your street address.

\date[Numerical Analysis and Programming] % (optional)
{\today}

\subject{Talks}
% This is only inserted into the PDF information catalog. Can be left
% out. 



% If you have a file called "university-logo-filename.xxx", where xxx
% is a graphic format that can be processed by latex or pdflatex,
% resp., then you can add a logo as follows:

% \pgfdeclareimage[height=0.5cm]{university-logo}{university-logo-filename}
% \logo{\pgfuseimage{university-logo}}



% Delete this, if you do not want the table of contents to pop up at
% the beginning of each subsection:
%\AtBeginSubsection[]
%{
%  \begin{frame}<beamer>{Outline}
%    \tableofcontents[currentsection,currentsubsection]
%  \end{frame}
%}


% If you wish to uncover everything in a step-wise fashion, uncomment
% the following command: 

%\beamerdefaultoverlayspecification{<+->}


\begin{document}

\begin{frame}
  \titlepage
\end{frame}

\begin{frame}{Outline}
  \tableofcontents
  % You might wish to add the option [pausesections]
\end{frame}


% Since this a solution template for a generic talk, very little can
% be said about how it should be structured. However, the talk length
% of between 15min and 45min and the theme suggest that you stick to
% the following rules:  

% - Exactly two or three sections (other than the summary).
% - At *most* three subsections per section.
% - Talk about 30s to 2min per frame. So there should be between about
%   15 and 30 frames, all told.
\section[Introduction]{Introduction}
\begin{frame}{Introduction}
\begin{itemize}
\item N-dimensional homogeneous arrays (ndarray) 
\item Universal functions (ufunc)
\begin{itemize}
\item built-in linear algebra, FFT, PRNGs 
\end{itemize}
\item Tools for integrating with C/C++/Fortran 
\item Heavy lifting done by optimized C/Fortran libraries
\begin{itemize}
\item ATLAS or MKL, UMFPACK, FFTW, etc...
\end{itemize} 
\end{itemize}
\end{frame}

\begin{frame}{Basics}
\begin{itemize}
\item NumPy's main object is the \alert{homogeneous multidimensional array}. 
\item This is a table of elements indexed by a tuple of positive integers.
\item  Arrays can have several dimensions or \alert{axes}. 
\item The number of axes will often be called \alert{rank}.
\end{itemize}
\end{frame}



\begin{frame}[fragile]
\frametitle{Attributes}
\begin{itemize}
\item {\tt ndarray.ndim}: the number of axes (dimensions) of the array, or the rank. 
\item {\tt ndarray.shape}: the dimensions of the array.  
\begin{itemize}
\item For a matrix with $n$ rows and $m$ columns, shape will be $(n,m)$. 
\item The length of the shape tuple is therefore the rank, or number of dimensions, $ndim$.
\end{itemize}
\item {\tt ndarray.size}: the total number of elements of the array.
\item {\tt ndarray.dtype}: an object describing the type of the elements in the array. 
\item {\tt ndarray.itemsize}: the size in bytes of each element of the array. 
\item {\tt ndarray.data} the buffer containing the actual elements of the array. 
\end{itemize}
\end{frame}
\end{document}


