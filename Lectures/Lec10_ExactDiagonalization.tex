% $Header: /home/vedranm/bitbucket/beamer/solutions/generic-talks/generic-ornate-15min-45min.en.tex,v 90e850259b8b 2007/01/28 20:48:30 tantau $
\documentclass{beamer}
%\documentclass[handout]{beamer}
\usefonttheme[onlymath]{serif}
% This file is a solution template for:
\usepackage{algorithm}
\usepackage{algpseudocode}
\usepackage{multicol}
% - Giving a talk on some subject.
% - The talk is between 15min and 45min long.
% - Style is ornate.



% Copyright 2004 by Till Tantau <tantau@users.sourceforge.net>.
%
% In principle, this file can be redistributed and/or modified under
% the terms of the GNU Public License, version 2.
%
% However, this file is supposed to be a template to be modified
% for your own needs. For this reason, if you use this file as a
% template and not specifically distribute it as part of a another
% package/program, I grant the extra permission to freely copy and
% modify this file as you see fit and even to delete this copyright
% notice. 

\mode<presentation>
{
  \usetheme{Warsaw}
  % or ...

  \setbeamercovered{transparent}
  % or whatever (possibly just delete it)
}
\setbeamertemplate{navigation symbols}{} 

\usepackage[english]{babel}
% or whatever

\usepackage[latin1]{inputenc}
% or whatever
\useoutertheme{default}

\usepackage{times}
\usepackage[T1]{fontenc}
% Or whatever. Note that the encoding and the font should match. If T1
% does not look nice, try deleting the line with the fontenc.
\newcommand{\beforeverb}{\footnotesize}
\newcommand{\afterverb}{\normalsize}

\title[Exact Diagonalization for Quantum Systems] % (optional, use only with long paper titles)
{Lecture 10}

\subtitle
{Exact Diagonalization for Quantum Systems } % (optional)

\author[Ying-Jer Kao] % (optional, use only with lots of authors)
{Ying-Jer Kao}
% - Use the \inst{?} command only if the authors have different
%   affiliation.

\institute[National Taiwan University] % (optional, but mostly needed)
{
  Department of Physics\\
 National Taiwan University
  }
% - Use the \inst command only if there are several affiliations.
% - Keep it simple, no one is interested in your street address.

\date[Numerical Analysis and Programming] % (optional)
{\today}

\subject{Talks}
% This is only inserted into the PDF information catalog. Can be left
% out. 



% If you have a file called "university-logo-filename.xxx", where xxx
% is a graphic format that can be processed by latex or pdflatex,
% resp., then you can add a logo as follows:

% \pgfdeclareimage[height=0.5cm]{university-logo}{university-logo-filename}
% \logo{\pgfuseimage{university-logo}}



% Delete this, if you do not want the table of contents to pop up at
% the beginning of each subsection:
%\AtBeginSubsection[]
%{
%  \begin{frame}<beamer>{Outline}
%    \tableofcontents[currentsection,currentsubsection]
%  \end{frame}
%}


% If you wish to uncover everything in a step-wise fashion, uncomment
% the following command: 

%\beamerdefaultoverlayspecification{<+->}


\begin{document}

\begin{frame}
  \titlepage
\end{frame}

\begin{frame}{Outline}
  \tableofcontents
  % You might wish to add the option [pausesections]
\end{frame}


% Since this a solution template for a generic talk, very little can
% be said about how it should be structured. However, the talk length
% of between 15min and 45min and the theme suggest that you stick to
% the following rules:  

% - Exactly two or three sections (other than the summary).
% - At *most* three subsections per section.
% - Talk about 30s to 2min per frame. So there should be between about
%   15 and 30 frames, all told.
\section[Introduction]{Introduction}

\begin{frame}{Introduction}
    \begin{itemize}
        \item Solving Schr\"{o}dinger equation for a manybody system for a finitie size system is an eigenvalue problem.
        \[
        H|\psi\rangle=E|\psi\rangle
        \]
        \item We can simulate all kind of models if we can diagonalize the Hamiltonian
        \item Limited by the size of the matrices we can store and diagonalize. 
    \end{itemize}
\end{frame}
\begin{frame}{Hilbert Space}
    \begin{itemize}
    \item Dimension of Hilbert space grows exponentially with system size.
    \item For a system with $N$ sites and $d$ states per site, the dimension of the Hilbert space is $d^N$.
    \item Example: Spin-1/2 chain with $N$ sites has a Hilbert space of dimension $2^N$.
 $$
|\psi\rangle=\sum_{\{S\}} a\left(s_1, s_2, \cdots, s_N\right)\left|s_1 s_2 \cdots, s_N\right\rangle
$$
basis states on each site: $\left|\uparrow\right\rangle$ and $\left|\downarrow\right\rangle$

    \end{itemize}
\end{frame}
\begin{frame}{Exponential Wall}
    \begin{itemize}
        \item $N=2$: basis states of the system
        \[
        \left|\uparrow_1\uparrow_2\right\rangle, \left|\uparrow_1\downarrow_2\right\rangle, \left|\downarrow_1\uparrow_2\right\rangle, \left|\downarrow_1\downarrow_2\right\rangle
        \]
        \item $N=40$, $2^40 \sim 10^{12}$ basis states, and the Hamiltonian has a dimension of $2^40\times 2^40$!!
        \item For a FP64 complex number storage, we need 16 bytes, we need \textbf{17.5}TB of memory to store the coefficients of the wavefunction!!
        \item Impossible to store the full Hamiltonian!!
    \end{itemize}
\end{frame}

\begin{frame}{State of the Art}

\begin{itemize}
   \item  Fractional quantum Hall effect\\
    different filling fractions ν, up to 16-20 electrons up to 300 million basis states, up to 1 billion in the near future
    \item Spin $S=1/2$ models:\\
  40 spins square lattice, 39 sites triangular, 42 sites star lattice at $S_z=0$, 64 spins or more in elevated magnetization sectors up to 1.5 billion(=$10^9$) basis states with symmetries, up to 4.5 billion without.
    \item $t$-$J$ models:\\
    32 sites checkerboard with 2 holes, 32 sites square lattice with 4 holes, up to 2.8 billion basis states
   \item Hubbard models
    20 sites square lattice at half filling, 20 sites quantum dot structure, 22-25 sites in ultracold atoms setting, up to 80 billion basis states.
  \item  \alert{low-lying eigenvalues}, not full diagonalization.
\end{itemize}    
\end{frame}
\section{Exact Diagonalization}
\begin{frame}{Exact Diagonalization Procesure}
    \begin{itemize}
        \item Coding the basis states.
        \item Reducing Hilbert space using symmetries.
        \item Finding the nonzero matrix entries of the Hamiltonian.
        \item Diagonalization.
        \item Calculating physical observable's expectation value.
    \end{itemize}
\end{frame}
\begin{frame}{Heisenberg Chain}
$$
        H=J \sum_{i=0}^{N-1} \mathbf{S}_i \cdot \mathbf{S}_{i+1}=J \sum_{i=0}^{N-1}\left[S_i^z S_{i+1}^z+\frac{1}{2}\left(S_i^{+} S_{i+1}^{-}+S_i^{-} S_{i+1}^{+}\right)\right]
$$
    \begin{itemize}
    \item Using the standard notation for basis states 
    $$
    \left|S_0^z S_1^z \ldots S_{N-1}^z\right\rangle,
    $$
    with $S_i^z$ from left to right corresponding to the spin states on the lattice site $i=0,1,2,\ldots, N-1$.
    \end{itemize}
\end{frame}
\begin{frame}{Basis States}
    \begin{itemize}
        \item We need to represent the \textbf{states} in the computer.
        \item Representation of basis states in terms of binary numbers: $S=1/2$
$$
\left.\left|\uparrow_1 \downarrow_2 \downarrow_3 \uparrow_4 \ldots\right\rangle\right) \Rightarrow\left[\begin{array}{llll}
1 & 0 & 0 & 1
\end{array}\right]_2=9
$$
\item Detection of up/down spins by bit-test
\item Transverse exchange $S^{+} S^{-}+S^{-} S^{+}$ can be done by an XOR operation
$$
\left[\begin{array}{llll}
1 & 0 & 0 & 1
\end{array}\right]_2 \oplus\left[\begin{array}{llll}
1 & 1 & 0 & 0
\end{array}\right]_2=\left[\begin{array}{llll}
0 & 1 & 0 & 1
\end{array}\right]_2=5
$$


    \end{itemize}

\end{frame}
\end{document}


