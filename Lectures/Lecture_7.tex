% $Header: /home/vedranm/bitbucket/beamer/solutions/generic-talks/generic-ornate-15min-45min.en.tex,v 90e850259b8b 2007/01/28 20:48:30 tantau $
%\documentclass{beamer}

\documentclass[handout]{beamer}

% This file is a solution template for:

% - Giving a talk on some subject.
% - The talk is between 15min and 45min long.
% - Style is ornate.



% Copyright 2004 by Till Tantau <tantau@users.sourceforge.net>.
%
% In principle, this file can be redistributed and/or modified under
% the terms of the GNU Public License, version 2.
%
% However, this file is supposed to be a template to be modified
% for your own needs. For this reason, if you use this file as a
% template and not specifically distribute it as part of a another
% package/program, I grant the extra permission to freely copy and
% modify this file as you see fit and even to delete this copyright
% notice. 

\mode<presentation>
{
  \usetheme{Warsaw}
  % or ...

  \setbeamercovered{transparent}
  % or whatever (possibly just delete it)
}
\setbeamertemplate{navigation symbols}{} 

\usepackage[english]{babel}
% or whatever

\usepackage[latin1]{inputenc}
% or whatever
\useoutertheme{default}

\usepackage{times}
\usepackage[T1]{fontenc}
% Or whatever. Note that the encoding and the font should match. If T1
% does not look nice, try deleting the line with the fontenc.
\newcommand{\beforeverb}{\footnotesize}
\newcommand{\afterverb}{\normalsize}

\title[File I/O and Exceptions] % (optional, use only with long paper titles)
{Lecture 7}

\subtitle
{File I/O and Exceptions} % (optional)

\author[Ying-Jer Kao] % (optional, use only with lots of authors)
{Ying-Jer Kao}
% - Use the \inst{?} command only if the authors have different
%   affiliation.

\institute[National Taiwan University] % (optional, but mostly needed)
{
  Department of Physics\\
 National Taiwan University
  }
% - Use the \inst command only if there are several affiliations.
% - Keep it simple, no one is interested in your street address.

\date[Numerical Analysis and Programming] % (optional)
{\today}

\subject{Talks}
% This is only inserted into the PDF information catalog. Can be left
% out. 



% If you have a file called "university-logo-filename.xxx", where xxx
% is a graphic format that can be processed by latex or pdflatex,
% resp., then you can add a logo as follows:

% \pgfdeclareimage[height=0.5cm]{university-logo}{university-logo-filename}
% \logo{\pgfuseimage{university-logo}}



% Delete this, if you do not want the table of contents to pop up at
% the beginning of each subsection:
%\AtBeginSubsection[]
%{
%  \begin{frame}<beamer>{Outline}
%    \tableofcontents[currentsection,currentsubsection]
%  \end{frame}
%}


% If you wish to uncover everything in a step-wise fashion, uncomment
% the following command: 

%\beamerdefaultoverlayspecification{<+->}


\begin{document}

\begin{frame}
  \titlepage
\end{frame}

\begin{frame}{Outline}
  \tableofcontents
  % You might wish to add the option [pausesections]
\end{frame}


% Since this a solution template for a generic talk, very little can
% be said about how it should be structured. However, the talk length
% of between 15min and 45min and the theme suggest that you stick to
% the following rules:  

% - Exactly two or three sections (other than the summary).
% - At *most* three subsections per section.
% - Talk about 30s to 2min per frame. So there should be between about
%   15 and 30 frames, all told.

\section[File I/O]{File I/O}
\subsection[File Objects]{File Objects}
\begin{frame}[fragile]
\frametitle{File Objects}
\begin{itemize}

\item The coimmand {\tt open()} returns a \alert{file object}.
\item It is most commonly used with two arguments: {\tt open(filename, mode)}.
\beforeverb
\begin{verbatim}
>>> fin = open('words.txt')
>>> print fin
<open file 'words.txt', mode 'r' at 0xb7f4b380>
>>> f = open('/tmp/workfile', 'w')
>>> print f 
<open file '/tmp/workfile', mode 'w' at 80a0960> 
\end{verbatim}
\afterverb
\item Mode \verb"'r'" indicates that this file is open for
reading and \verb"'w'" for writing.

\end{itemize}

\end{frame}
\begin{frame}[fragile]
\frametitle{File Objects Methods}
\begin{itemize}
\item The file object provides several methods for reading.
\item The method {\tt readline}  reads characters from the file
until it gets to a \alert{newline} and returns the result as a
string:
\beforeverb
\begin{verbatim}
>>> fin.readline()
'aa\r\n'
\end{verbatim}
\afterverb
%
\item  \verb"\r\n" represents two whitespace characters,
a \alert{carriage return} and a \alert{newline}, that separate this word from the
next.
\item If {\tt readline} is called again again, one gets the next line:
\beforeverb
\begin{verbatim}
>>> fin.readline()
'aah\r\n'
\end{verbatim}
\afterverb
\item The whitespace characters can be removed by the string method {\tt strip}
\beforeverb
\begin{verbatim}
>>> line = fin.readline()
>>> word = line.strip()
>>> print word
aah
\end{verbatim}
\afterverb
\end{itemize}
\end{frame}


\begin{frame}[fragile]
\frametitle{Loop Iteration}
\begin{itemize}
\item {\tt f.readlines()} returns a list containing all the lines of data in the file.  Only complete lines will be returned.
\beforeverb
\begin{verbatim}
>>> f.readlines()
['First line.\n', 'Second line.\n']
\end{verbatim}
\afterverb
\item One can also use a file object as part of a {\tt for} loop.
\item This program reads {\tt words.txt} and prints each word, one
per line:
\beforeverb
\begin{verbatim}
fin = open('words.txt')
for line in fin:
    word = line.strip()
    print word
\end{verbatim}
\afterverb
%
\end{itemize}
\end{frame}
\begin{frame}[fragile]
\frametitle{Write to File}
\begin{itemize}
\item The {\tt write} method puts data into the file.

\beforeverb
\begin{verbatim}
>>> line1 = "This here's the wattle,\n"
>>> fout.write(line1)
\end{verbatim}

\afterverb
\item If {\tt write} is called again, it adds the new data to the end of the file.
\item When all the input and output are done, close the file.

\beforeverb
\begin{verbatim}
>>> fout.close()
\end{verbatim}
\end{itemize}
\end{frame}

\begin{frame}[fragile]
\frametitle{Formatting Operator}
\begin{itemize}
\item Sometimes it is useful to print the output in a user-defined format instead of the default format. 
\item Whe the first operand is a string, the operator \verb!%!  is the \alert{format operator}. 
\beforeverb
\begin{verbatim}
>>> camels = 42
>>> 'I have spotted %d camels.' % camels
'I have spotted 42 camels.'
\end{verbatim}
\afterverb

\item The first operand is the \alert{format string}, which contains one or more \alert{format sequences}, which
specify how the second operand is formatted.  
\item If there is more than one format sequence in the string,
the second argument has to be a \alert{tuple}. 
\beforeverb
\begin{verbatim}
>>> dogs = 42
>>> cats =12
>>> 'I have spotted %d dogs and %d cats' % (dogs, cats)
'I have spotted 42 dogs and 12 cats.'
\end{verbatim}
\afterverb

\end{itemize}
\end{frame}

\begin{frame}[fragile]
\frametitle{Filenames and Paths}
\begin{itemize}
\item The {\tt  os} module provides several useful functions to obtain informations about filenames, directories and paths. 
\begin{block}{Example}

\beforeverb
\begin{verbatim}
def walk(dir):
    for name in os.listdir(dir):
        path = os.path.join(dir, name)

        if os.path.isfile(path):
            print path
        else:
            walk(path)
\end{verbatim}
\end{block}
\end{itemize}
\end{frame}


\begin{frame}[fragile]
\frametitle{Modules}
\begin{itemize}
\item  Any file that contains Python code can be imported as a module.


\beforeverb
\begin{verbatim}
def linecount(filename):
    count = 0
    for line in open(filename):
        count += 1
    return count

print linecount('wc.py')
\end{verbatim}
\afterverb
\end{itemize}
\end{frame}

\begin{frame}[fragile]
\frametitle{Module}
\begin{itemize}
\item When this program is run, it reads itself and prints the number
of lines in the file, 7.

\item This module can also be imported.
\beforeverb
\begin{verbatim}
>>> import wc
7
\end{verbatim}
\afterverb
\item
This  provides a function called \verb"linecount":

\beforeverb
\begin{verbatim}
>>> wc.linecount('wc.py')
7
\end{verbatim}
\afterverb
%
\item 
The test code at the bottom is always executed. 
\end{itemize}
\end{frame}


\begin{frame}[fragile]
\frametitle{Executing Modules as Scripts}
\begin{itemize}
\item Normally, we expect when the module is \alert{imported}, the functions are \alert{defined}, but not \alert{executed}.
\item  Programs that will be imported as modules often use the following idiom:
\beforeverb
\begin{verbatim}
if __name__ == '__main__':
    print linecount('wc.py')
\end{verbatim}
\afterverb
%
\item 
\verb"__name__" is a built-in variable that is set when the
program starts.  
\item If the program is running as a script,
\verb"__name__" has the value \verb"__main__"; in that
case, the test code is executed.  
\item If the module is being imported, the test code is skipped.
\end{itemize}
\end{frame}
\section[Exceptions]{Exceptions}
\begin{frame}[fragile]
\frametitle{Exceptions}
 \alert{Runtime errors} can occur during file I/O.
\begin{itemize}
\item  File doesn't exist
\beforeverb
\begin{verbatim}
>>> fin = open('bad_file')
IOError: [Errno 2] No such file or directory: 'bad_file'
\end{verbatim}
\afterverb
%
\item No permission to access a file:

\beforeverb
\begin{verbatim}
>>> fout = open('/etc/passwd', 'w')
IOError: [Errno 13] Permission denied: '/etc/passwd'
\end{verbatim}
\afterverb
%
\item  Open a directory for reading:
\beforeverb
\begin{verbatim}
>>> fin = open('/home')
IOError: [Errno 21] Is a directory
\end{verbatim}
\afterverb
\end{itemize}
\end{frame}

\begin{frame}[fragile]
\frametitle{Exceptions}
\begin{itemize}
\item Errors detected during execution are called \alert{exceptions} and are not unconditionally fatal.
\item Exceptions come in different \alert{types}, and the type is printed as part of the message. 
\item The string printed as the exception type is the name of the built-in exception that occurred.
\scriptsize
\begin{verbatim}
>>> 10 * (1/0)
Traceback (most recent call last):
  File "<stdin>", line 1, in ?
ZeroDivisionError: integer division or modulo by zero
>>> 4 + spam*3
Traceback (most recent call last):
  File "<stdin>", line 1, in ?
NameError: name 'spam' is not defined
>>> '2' + 2
Traceback (most recent call last):
  File "<stdin>", line 1, in ?
TypeError: cannot concatenate 'str' and 'int' objects
\end{verbatim}
\afterverb
\end{itemize}
\end{frame}

\begin{frame}[fragile]
\frametitle{Handling Exceptions}
\begin{itemize}
\item It is possible to handle exceptions in the program using the {\tt try} and {\tt except} statements. 
\item Handling an exception with a {\tt try} statement is called \alert{
catching} an exception.  
\item In general,
catching an exception is useful to fix the problem, or try
again, or at least end the program gracefully.

\beforeverb
\begin{verbatim}
try:    
    fin = open('bad_file')
    for line in fin:
        print line
    fin.close()
except:
    print 'Something went wrong.'
\end{verbatim}
\afterverb
%


\end{itemize}
\end{frame}

\begin{frame}[fragile]
\frametitle{{\tt try} statement}
The {\tt try} statement works as follows. 

\begin{itemize}
\item First, the {\tt try} clause (the statement(s) between the {\tt try} and {\tt except} keywords) is executed. 
\item If no exception occurs, the {\tt except} clause is skipped and execution of the {\tt try} statement is finished.
\item If an exception occurs during execution of the {\tt try} clause, the rest of the clause is skipped. Then if its type matches the exception named after the except keyword, the except {\tt clause} is executed, and then execution continues after the {\tt try} statement.
\item  If an exception occurs which does not match the exception named in the except clause, it is passed on to outer try statements; if no handler is found, it is an \alert{unhandled exception} and execution stops with an error message.
\end{itemize}
\end{frame}
\begin{frame}[fragile]
\frametitle{ Raising Exceptions}
\begin{itemize}
\item The {\tt raise} statement allows the programmer to force a specified exception to occur. 

\item This is  useful for checking if the input values are valid.
\beforeverb
\begin{verbatim}
def add_time(t1, t2):
    if not valid_time(t1) or not valid_time(t2):
        raise ValueError, 'invalid Time'
    seconds = time_to_int(t1) + time_to_int(t2)
    return int_to_time(seconds)
\end{verbatim}
\afterverb
%
\end{itemize}
\end{frame}
\begin{frame}[fragile]
\frametitle{The {\tt assert} statment}
\begin{itemize}
\item It  can also be achieved by the {\tt assert} statement, which checks a given condition
and raises an \alert{exception} if it fails:
\scriptsize
\begin{verbatim}
def add_time(t1, t2):
    assert valid_time(t1) and valid_time(t2), 'Invalid Time'
    seconds = time_to_int(t1) + time_to_int(t2)
    return int_to_time(seconds)
\end{verbatim}
\afterverb
%
\item {\tt assert} statements are useful because they distinguish
code that deals with normal conditions from code
that checks for errors.

\end{itemize}
\end{frame}
\end{document}


