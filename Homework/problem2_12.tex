\documentclass[12pt]{article}
\usepackage{times}
\usepackage{amsmath}
\usepackage{a4}
\usepackage{graphicx}
\usepackage{subfigure}
\usepackage{hyperref}

%\setlength{\textwidth}{6in}
\setlength{\textheight}{9in}
\pagestyle{empty}
%\setlength{\oddsidemargin}{0.5in}
%\setlength{\evensidemargin}{-0.5in}
%\setlength{\textwidth}{in}
\begin{document}

\begin{center}
\Large
\textbf{Computational Manybody Physics}\\
\large
Problem Set \#2\\
Due: Oct 25, in class

\end{center}
In this problem set, we will implement an exact-diagonalization for the frustrated Heisenberg chain, 
\[
H=J_1\sum_{\langle ij\rangle} \mathbf{S}_i \cdot \mathbf{S}_j +J_2 \sum_{\langle\langle ij\langle\rangle} \mathbf{S}_i \cdot \mathbf{S}_j ,
\]
based on the exact-diagonalization source code for the Heisenberg chain.\footnote{ See Prof. Anders Sandvik's website \textit{http://physics.bu.edu/~sandvik/vietri/index.html}}. 

\begin{enumerate} 
\item Using Lanczos method to find the energy of the ground state and lowest excited singlet ($S=0$) and triplet ($S=1$) for $L=4, 8, 16, 20$. Locate the critical point $g=J_2/J_1=g_c$ by the level crossing of the excited states.
\item Compute the spin correlation functions for $g=0,g_c,0.40,0.45,0.50$. Plot $C(N/2)(N/2)$ vs $1/N$.
\item Compute the dimer correlation functions for $g=0,g_c,0.40,0.45,0.50$. Plot $D(N/2)-D(N/2-1)$ vs $1/N$.
\item Show that at the Majumdar-Ghosh point $g=0.5$, $D(N/2)-D(N/2-1)$ is size independent.
\end{enumerate}
\end{document}
