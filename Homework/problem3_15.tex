\documentclass[12pt]{article}
\usepackage{times}
\usepackage{amsmath}
\usepackage{a4}
\usepackage{graphicx}
\usepackage{subfigure}
\usepackage[colorlinks=true]{hyperref}

%\setlength{\textwidth}{6in}
\setlength{\textheight}{9in}
\pagestyle{empty}
%\setlength{\oddsidemargin}{0.5in}
%\setlength{\evensidemargin}{-0.5in}
%\setlength{\textwidth}{in}
\begin{document}

\begin{center}
\Large
\textbf{Computational Manybody Physics}\\
\large
Problem Set \#3\\


\end{center}

In this problem set, we will learn the basics of DMRG code using the tutorials from
 \href{http://dmrg101_tutorial.readthedocs.org/en/latest/index.html}{DMRG101}.
\begin{enumerate}
\item Install the necessary packages following the instruction  \href{https://sites.google.com/site/dmrg101/tutorial}{here}.
\item Following the instruction described in the tutorial, finish all the exercises for the spin systems. Analyze your data using the ipython notebook.
\end{enumerate}

\end{document}
