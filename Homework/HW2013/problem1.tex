\documentclass[12pt]{article}
\usepackage{times}
\usepackage{a4}
%\usepackage{algorithm}
\usepackage{algpseudocode}
\usepackage{amssymb}
\usepackage{amsmath}
\usepackage{graphicx}
\usepackage{hyperref}
%\setlength{\oddsidemargin}{-0.5in}
%\setlength{\evensidemargin}{-0.5in}
%\setlength{\textwidth}{in}
\begin{document}
\begin{center}
\Large
\textbf{Numerical Analysis and Programming}\\
\large
Problem Set \#1\\
Due: March 21
\end{center}
\begin{enumerate}
\item A prime number is a natural number greater than 1 that has no positive divisors other than 1 and itself. The simplest algorithm to check if a natural number $y$ is a prime number can be described by the pseudocode:
\begin{algorithmic}[1]

\For {$i \gets 1, y$}
\If {$ y=0 \mod{i}$}
\State \Return $y$ is not prime
\EndIf
\EndFor
\State \Return $y$ is prime
\end{algorithmic}

\begin{itemize}
\item The algorithm works but is not efficient. Explore ways to improve the efficiency of the algorithm, and  write a program to find the  1000th prime number.
\end{itemize}
\item  A result from number theory  states that for sufficiently large $n$, the product of
the primes less than $n$ is less than or equal to $e^n$ and that as $n$ grows, this becomes a tight
bound (that is, the ratio of the product of the primes to $e^n$ gets close to 1 as $n$ grows).
\begin{itemize}
\item Write a program that computes the sum of the logarithms of all the primes from 2 to some
number $n$, and output the sum of the logs of the primes, the number $n$, and the ratio of these
two quantities. Test this for different values of $n$.
\end{itemize}
\end{enumerate}
To finish homework, you need to know two statements:\texttt{for} and \texttt{if}. Please read the book and  google for information.


Hand-In Procedure

\begin{enumerate}
\item Please write two problems in a \textbf{single} .py file and summit the executable .py file. Print $1000^{th}$ prime number first.
\item Follow the format of following picture.( use the \textbf{template})
\item Summit the file to the website at \url{http://gandalf.phys.ntu.edu.tw}
\item You have \textbf{ten} chances to summit your answer and the score of your last answer will
  be your final score.
\item Please go to the website for more details.
\end{enumerate}
\includegraphics[width=0.8\textwidth]{hw1exp.png}

Samples for problem 1\\

\begin{minipage}{0.5\textwidth}
Sample input:\\
5\\
666\\
1984\\
140112\\
\end{minipage}
\begin{minipage}{0.5\textwidth}
Sample output:\\
3.40119738166 5 0.680239476332\\
640.374463324 666 0.961523218204\\
1909.44761907 1984 0.962423195097\\
139714.259984 140112 0.997161270869\\

\end{minipage}
\\
\\
Note: Discussions are strongly encouraged, but \textbf{plagiarism} is strictly forbidden. All parties involved in copying will get \textbf{zero} for their homework.
\end{document}
