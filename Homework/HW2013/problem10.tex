\documentclass[12pt]{article}
\usepackage{times}
\usepackage{amsmath}
\usepackage{a4}
\usepackage{graphicx}
%\setlength{\oddsidemargin}{-0.5in}
%\setlength{\evensidemargin}{-0.5in}
%\setlength{\textwidth}{in}
\pagestyle{empty}
\begin{document}
\begin{center}
\Large
\textbf{Numerical Analysis and Programming}\\
\large
Problem Set \#10\
Due: May 24
\end{center}
This homework should be completed by a \textbf{joint effort of two groups}. 
\begin{enumerate}

\item Implement a \verb!Radar_Array! class which has two radar stations $A$ and $B$ modeled by the \verb!Radar! class. The two stations are  separated by a distance $d$ to track a moving object and obtain the current velocity and position of the object. 

\item Write a program to implement a spacecraft moving at a trajectory described by 
\[
x=4t; y=100+3t-5t^2.
\]
A radar array tracks its position and velocity. Use this information to update the spacecraft's velocity information. 

\begin{figure}[b]
\begin{center}
\includegraphics[width=3in]{HW10_Fig1}
\end{center}
\end{figure}

\end{enumerate}
Hand-In Procedure  
\begin{enumerate}
\item Save your code as  ps10.py.
 Do not ignore this step or save your file(s) with different names. 
\item Time and Collaboration Info \\
At the start of each file, in a comment, write down the number of hours (roughly) you spent on 
the problems in that part, and the names of the people you collaborated with. For example:
\begin{verbatim}
# Problem Set 10
# Student ID: r99222024
# Collaborators' ID: r99222024
# Time: 00:10
... your code goes here ...
\end{verbatim}
\item Upload to Ceiba.
\end{enumerate}
Note: Discussions are strongly encouraged, but \textbf{no copying} is allowed. All parties involved in copying will get \textbf{zero} for their homework. 
\end{document}
