\documentclass[12pt]{article}
\usepackage{times}
\usepackage{amsmath}
\usepackage{a4}
\usepackage{graphicx}
%\setlength{\oddsidemargin}{-0.5in}
%\setlength{\evensidemargin}{-0.5in}
%\setlength{\textwidth}{in}
\pagestyle{empty}
\begin{document}
\begin{center}
\Large
\textbf{Numerical Analysis and Programming}\\
\large
Problem Set \#9\
Due: May 17
\end{center}

\begin{enumerate}

\item We will start building elements for use in our final project. 
 Use the class definition in Lab9 implement the associated methods for the {\tt Spacecraft} class.
\item Write a function {\tt isCollision} which checks if two spacecrafts collide. (Notice this is not a class method)
\item Write a function {\tt accelerateShip} which change the velocity of the spacecraft. 
\end{enumerate}
Hand-In Procedure  
\begin{enumerate}
\item Save your code as  ps9.py.
 Do not ignore this step or save your file(s) with different names. 
\item Time and Collaboration Info \\
At the start of each file, in a comment, write down the number of hours (roughly) you spent on 
the problems in that part, and the names of the people you collaborated with. For example:
\begin{verbatim}
# Problem Set 9
# ID: b982020xx
# Collaborators' ID: b982020xx
# Time: 3:30
# 
... your code goes here ...
\end{verbatim}
\item Upload to Ceiba.
\end{enumerate}
Note: Discussions are strongly encouraged, but \textbf{no copying} is allowed. All parties involved in copying will get \textbf{zero} for their homework. 
\end{document}
