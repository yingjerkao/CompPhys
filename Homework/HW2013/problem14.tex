\documentclass[12pt]{article}
\usepackage{times}
\usepackage{amsmath}
\usepackage{a4}
\usepackage{graphicx}
%\setlength{\oddsidemargin}{-0.5in}
%\setlength{\evensidemargin}{-0.5in}
%\setlength{\textwidth}{in}
\pagestyle{empty}
\begin{document}
\begin{center}
\Large
\textbf{Numerical Analysis and Programming}\\
\large
Problem Set \#14\
Due: June 16
\end{center}

\begin{enumerate}
\item Implement a module in which the fourth order Runge-Kutta method is implemented.  
\end{enumerate}
Hand-In Procedure  
\begin{enumerate}
\item Save your code as py14.py.
 Do not ignore this step or save your file(s) with different names. 
\item Time and Collaboration Info \\
At the start of each file, in a comment, write down the number of hours (roughly) you spent on 
the problems in that part, and the names of the people you collaborated with. For example:
\begin{verbatim}
# Problem Set 14
# Name: Ying-Jer Kao
# Collaborators: Alice Lee
# Time: 3:30
# 
... your code goes here ...
\end{verbatim}
\item Upload to Ceiba.
\end{enumerate}
Note: Discussions are strongly encouraged, but \textbf{no copying} is allowed. All parties involved in copying will get \textbf{zero} for their homework. 
\end{document}
