\documentclass[12pt]{article}
\usepackage{times}
\usepackage{amsmath}
\usepackage{a4}
\usepackage{graphicx}
\usepackage{hyperref}
\usepackage{algpseudocode}
%\setlength{\oddsidemargin}{-0.5in}
%\setlength{\evensidemargin}{-0.5in}
%\setlength{\textwidth}{in}
\begin{document}
\begin{center}
\Large
\textbf{Numerical Analysis and Programming}\\
\large
Problem Set \#2\\
Due: April 4
\end{center}
\begin{enumerate}
%\item Another way to solve the number of spiders, rabbits and chickens for given numbers of heads and legs is to use the \texttt{solve} function which solves the problem with only rabbits and chickens. Write a function \texttt{solve3} to obtain all the solutions for given heads and legs. (Save as \textit{ps4a.py})
%\small
%\begin{verbatim}
%def solve(numHeads, numLegs):
%    for numChickens in range(numHeads+1): 
%        numRabbits=numHeads-numChickens
%        totalLegs=4*numRabbits+2*numChickens
%        if totalLegs==numLegs:
%            return [numRabbits,numChickens]
%    return [None, None]
%\end{verbatim}
\item
A simple sorting algorithm called \href{http://en.wikipedia.org/wiki/Bubble_sort}{\textit{bubble sort}}. Repeatedly through a list to be sorted, compare each pair and switch them if they are wrong ordered. Described by the following pseudocode

\begin{algorithmic}[1]
\Function{bubbleSort}{$A: array_like$}
\Repeat
\State swapped = false
\For {$i = 1$ to $length(A) - 1$ inclusive}
\If{$A[i-1] > A[i]$}
\State swap( A[i-1], A[i] )
\State swapped = true
\EndIf
\EndFor

\Until{not swapped}
end procedure
\EndFunction
%\State{()}
%\State{$A[i]$:True, $i=2,\ldots,n$} 
%
% 
%\For {$i \gets 2, 3, 4, \ldots, \sqrt{n} $}
%\If {$ A[i]==$ true}
%\For{$j \gets  i^2, i^2+i, i^2+2i,\ldots, n$}
%      \State{$A[j] \gets $ false}
%\EndFor
%\EndIf
%\EndFor
%\For {$i \gets 2, \ldots n$}
%\If {$A[i]$==True}
%\State{ $i$ is a prime}
%\EndIf
%\EndFor
\end{algorithmic}
Write a function called \texttt{bubble\_sort(a)}  based on the above pseudocode. 

\item 
The greatest common divisor (GCD) of $a$ and $b$ is the largest number
that divides both of them with no remainder.
One way to find the GCD of two numbers is Euclid's algorithm,
which is based on the observation that if $r$ is the remainder
when $a$ is divided by $b$, then $gcd(a, b) = gcd(b, r)$.
As a base case, we can consider $gcd(a, 0) = a$.
Write two functions called
\verb"gcd_recursive(a,b)" and \verb"gcd_iteration(a,b)" that takes parameters {\tt a} and {\tt b}
and returns their greatest common divisor using the \textit{recursive} and \textit{iteration} methods.
\item $\pi$ can be approximated by the following algorithm which calculates the perimeters of polygons inscribing and circumscribing a circle, starting with hexagons, and successively doubling the number of sides.
\begin{eqnarray}
t_0&=\frac{1}{\sqrt{3}}\nonumber\\
t_{i+1}&=\frac{\sqrt{t_i^2+1}-1}{t_i}\label{form1}\\
          &=\frac{t_i}{\sqrt{t_i^2+1}+1}\label{form2}
\end{eqnarray}
and $\pi \sim 6 \times 2^i\times t_i$.\\
Write a function called \texttt{pi\_approx(n, i)} to calculate $\pi$ using both Eqs.~(\ref{form1}) and (\ref{form2}). Where $n$ means the n-th approximation, and $i$ means which equation you used. Write another function called \texttt{pi\_error(i)} to determine for which the number of sides $n$, the round-off error blows up for each equation.

\end{enumerate} 

\textbf{Hand-In Procedure}  
 \begin{enumerate}
\item Please write two problems in a \textbf{single} \texttt{.py} file and summit the executable \texttt{.py} file. 
\item Follow the format of following picture.  Please use the provided \textbf{template}.
\item Summit the file to the website at \url{http://gandalf.phys.ntu.edu.tw}
\item You have \textbf{ten} chances to summit your answer and the score of your last answer will
  be your final score.
\item Please go to the website for more details.
\end{enumerate}

\includegraphics[width=1.0\textwidth]{prob2.jpg}

Note: Discussions are strongly encouraged, but \textbf{plagiarism} is strictly forbidden. All parties involved in copying will get \textbf{zero} for their homework.

\end{document}
