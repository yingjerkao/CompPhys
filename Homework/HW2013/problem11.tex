\documentclass[12pt]{article}
\usepackage{times}
\usepackage{amsmath}
\usepackage{a4}
\usepackage{graphicx}
%\setlength{\oddsidemargin}{-0.5in}
%\setlength{\evensidemargin}{-0.5in}
%\setlength{\textwidth}{in}
\pagestyle{empty}
\begin{document}
\begin{center}
\Large
\textbf{Numerical Analysis and Programming}\\
\large
Problem Set \#11\
Due: May 31
\end{center}
This homework should be turned in as  a two-people group.
\begin{enumerate}
\item Implement a \verb!Base! class such that it will take the trajectory data of a moving object and approximate it with a polynomial of degree $n$ using curve fitting. Use this polynomial to extrapolate the trajectory and perform interception using a missile. 
\item Complete the implementation of the test code, and perform some tests to make sure your code works. 
\item  Activity for the large group: In the same spirit as the test code, write a program to add radar arrays from problem set 10 and feed a trajectory  to the radars, and relay it to the base. \textbf{You don't have to turn in this part}, but it will help you develop a  program for the final project.  
\end{enumerate}

\begin{figure}[th]
\begin{center}
\includegraphics[width=4in]{HW11_Fig1}
\end{center}
\end{figure}

Hand-In Procedure  
\begin{enumerate}
\item Save your code as py11.py.
 Do not ignore this step or save your file(s) with different names. 
\item Time and Collaboration Info \\
At the start of each file, in a comment, write down the number of hours (roughly) you spent on 
the problems in that part, and the names of the people you collaborated with. For example:
\begin{verbatim}
# Problem Set 11
# Student ID: r99222024
# Collaborators' ID: r99222024
# Time: 00:10
# 
... your code goes here ...
\end{verbatim}
\item Upload to Ceiba.
\end{enumerate}
Note: Discussions are strongly encouraged, but \textbf{no copying} is allowed. All parties involved in copying will get \textbf{zero} for their homework. 
\end{document}
