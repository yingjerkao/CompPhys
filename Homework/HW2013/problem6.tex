\documentclass[12pt]{article}
\usepackage{times}
\usepackage{amsmath}
\usepackage{amssymb}
\usepackage{a4}
\usepackage{graphicx}
%\setlength{\oddsidemargin}{-0.5in}
%\setlength{\evensidemargin}{-0.5in}
%\setlength{\textwidth}{in}
\pagestyle{empty}
\begin{document}
\begin{center}
\Large
\textbf{Numerical Analysis and Programming}\\
\large
Problem Set \#6\\
Due: April 19
\end{center}
A palindrome is a word that is spelled the same backward and
forward, like ``noon'' and ``redivider''.  Recursively, a word
is a palindrome if the first and last letters are the same
and the middle is a palindrome. 

\begin{enumerate} 
\item Use Horner's algorithm together with Newton's and bisection methods  to find all  the \textbf{real} roots of a generic fifth-order polynomial:
\[
x^5+a_4x^4+a_3x^3+a_2x^2+a_1x+a_0=0,
\]
where $a_i\in \mathbb{R}$.

\item Use the function \verb"is_palindrome" written in Lab 5 to write a function \verb"has_palindrome" defined as 
\begin{verbatim}
def has_palindrome(s, start, len):
\end{verbatim}
which determines if the  substring of string \verb!s! starting from \verb!start! with length \verb!len! is palindrome, and  use\verb"has_palindrome" to solve the following puzzler:
\begin{quote}
``I was driving on the highway the other day and I happened to
notice my odometer. Like most odometers, it shows six digits,
in whole miles only. So, if my car had 300,000
miles, for example, I'd see 3-0-0-0-0-0.

``Now, what I saw that day was very interesting. I noticed that the
last 4 digits were palindromic; that is, they read the same forward as
backward. For example, 5-4-4-5 is a palindrome, so my odometer
could have read 3-1-5-4-4-5.

``One mile later, the last 5 numbers were palindromic. For example, it
could have read 3-6-5-4-5-6.  One mile after that, the middle 4 out of
6 numbers were palindromic.  And you ready for this? One mile later,
all 6 were palindromic!

``The question is, what was on the odometer when I first looked?''
\end{quote}
Write a function \verb!puzzle_solver()! which  tests all six-digit numbers and return the possible numbers that satisfy these requirements.
\end{enumerate} 
\newpage
Hand-In Procedure  
\begin{enumerate}
\item Use the provided template \textit{ps6.py}. 
 Do not  save your file with different names. 
\item Time and Collaboration Info \\
At the start of each file, in a comment, write down the number of hours (roughly) you spent on 
the problems in that part, and the names of the people you collaborated with. For example:
\begin{verbatim}
# Problem Set 6
# ID: b982020xx
# Collaborators' ID: b982020xx
# Time: 3:30
# 
... your code goes here ...
\end{verbatim}
\item Upload to Ceiba.
\end{enumerate}
Note: Discussions are strongly encouraged, but \textbf{no copying} is allowed. All parties involved in copying will get \textbf{zero} for their homework. 
\end{document}
