\documentclass[12pt]{article}
\usepackage{times}
\usepackage{amsmath}
\usepackage{a4}
\usepackage{graphicx}
%\setlength{\oddsidemargin}{-0.5in}
%\setlength{\evensidemargin}{-0.5in}
%\setlength{\textwidth}{in}
\begin{document}
\begin{center}
\Large
\textbf{Numerical Analysis and Programming}\\
\large
Problem Set \#1\\
Due: March 1, on-line
\end{center}
\begin{enumerate}
\item We introduced three different binary representations of signed integers: sign-and-magnitude, one's complement, and two's complement. In two's complement representation, addition of two integers is exactly the same as addition of their binary number representation; while in one's complement, to obtain the correct answer, an extra rule of adding the \textit{carry}  to the conventional binary addition is necessary. Please find the rule(s) one needs to apply in addition to the conventional binary arithmetics for adding two integers represented  in the sign-and-magnitude representation. 
\end{enumerate}
Please type up your solution and upload a pdf version to ceiba. 

Note: Discussions are strongly encouraged, but \textbf{no copying} is allowed. All parties involved in copying will get \textbf{zero} for their homework. 
\end{document}
