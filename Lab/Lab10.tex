\documentclass[12pt]{article}
\usepackage{times}
\usepackage{amsmath}
\usepackage{a4}
\usepackage{graphicx}
\setlength{\oddsidemargin}{-0.2in}
%\setlength{\evensidemargin}{-0.5in}
\setlength{\textwidth}{6.5in}
\begin{document}
\begin{center}
\Large
\textbf{Numerical Analysis and Programming}\\
\large
Lab Worksheet \#10\\

\end{center}

\begin{enumerate}
\item In final presentation, we will use something like this: \
First, we know that the gravity is the constant $g$, and the constant of air resistance is $b$. Both $g\approx10.0$ and $b\approx0.1$ are unknown. 
\begin{equation}
\left\{\begin{array}{ll}\frac{d^2x}{dt^2}=-b\frac{dx}{dt}\\
						\frac{d^2y}{dt^2}=-b\frac{dy}{dt}-g
	   \end{array} \right.
\end{equation}
\begin{equation}
\left\{\begin{array}{ll}\frac{dx}{dt}=v_{x0}e^{-bt}\\
						\frac{dy}{dt}=(v_{y0}+\frac{g}{b})e^{-bt}-\frac{g}{b}
	   \end{array} \right.
\end{equation}
\begin{equation}
\left\{\begin{array}{ll}x=-\frac{v_{x0}}{b}e^{-bt}+\frac{v_{x0}}{b}\\
						y=-\frac{(v_{y0}+\frac{g}{b})}{b}e^{-bt}-\frac{g}{b}t+\frac{(v_{y0}+\frac{g}{b})}{b}
	   \end{array} \right.
\end{equation}
At origin, if you shoot something in the air with velocity $v_0=15m/s$ angle $a=45\deg$, then you got the data in \verb'data_lab10.dat' in form \verb't x y'. Now you can use function \verb'curve_fit()' to get $g$ and $b$.\\
You can also compare the data and the fitting function by matplotlib.
%
%\begin{scriptsize}
%\begin{verbatim}
%def fx(t, b):
%	return -vx0/b*np.exp(-t*b)+vx0/b
%bb, = curve_fit(fx, ts, xs, p0=[.1])[0]
%print bb
%def fy(t, g):
%	return -(vy0+g/bb)/bb*np.exp(-bb*t) + (vy0+g/bb)/bb - g/bb*t
%	
%
%\end{verbatim}
%\end{scriptsize}
\end{enumerate}
\end{document}	
