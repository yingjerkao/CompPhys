\documentclass[12pt]{article}
\usepackage{times}
\usepackage{amsmath}
\usepackage{a4}
\usepackage{graphicx}
\usepackage{CJK}
\setlength{\oddsidemargin}{-0.2in}
%\setlength{\evensidemargin}{-0.5in}
\setlength{\textwidth}{6.5in}
\begin{document}
\begin{center}
\Large
\textbf{Numerical Analysis and Programming}\\
\large
Lab Worksheet \#2
\end{center}
\begin{enumerate}
\item When defining a function, it is possible to specify a \textit{default value} for one or more arguments. This creates a function that can be called with fewer arguments than it is defined to allow. For example,
\small
\begin{verbatim}
def ask_ok(prompt, retries=4, complaint='Yes or no, please!'): 
    while True:
        ok = raw_input(prompt) 
        if ok in ('y', 'ye', 'yes'):
            return True 
        if ok in ('n', 'no', 'nop', 'nope'):
            return False 
        retries = retries - 1 
        if retries < 0:
            raise IOError('refuse user') 
        print complaint
\end{verbatim}
\normalsize
Try the following possibilities and find the outputs.

\begin{itemize}\small
\item \verb!ask_ok('Do you really want to quit?')!
\item \verb!ask_ok('OK to overwrite the file?', 2)!
\item \verb"ask_ok('OK to overwrite the file?', 2, 'Come on, only yes or no!')"
\end{itemize}
\item Using the following examples to understand the scope ideas.
\begin{verbatim}
x=99
def func(y):
    z=x+y
    return z
\end{verbatim}
Run the command 
\begin{verbatim}
>>> print x, func(x)
\end{verbatim}
Explain how Python gets the results, and which variables are local to \verb"func". 
A slight modification of the code gives, 
\begin{verbatim}
x=99
def func(y):
    x=1
    z=x+y
    return z
\end{verbatim}
Run the \verb!print! command again. Explain how Python gets the results, and which variables are local to \verb"func".
\end{enumerate}
\end{document}	
