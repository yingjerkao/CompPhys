\documentclass[12pt]{article}
\usepackage{times}
\usepackage{amsmath}
\usepackage{a4}
\usepackage{graphicx}
\setlength{\oddsidemargin}{-0.2in}
\pagestyle{empty}
\usepackage{hyperref}
%\setlength{\evensidemargin}{-0.5in}
\setlength{\textwidth}{6.5in}
\begin{document}
\begin{center}
\Large
\textbf{Numerical Analysis and Programming}\\
\large
Lab Worksheet \#4
\end{center}
\begin{enumerate}
\item The following function approximately determines the \textit{machine epsilon} ($\epsilon$) for a given type (defaults to \verb!float!), using the definition that $\epsilon$ is the smallest positive number such that $1+\epsilon\ne 1$. 
\small
\begin{verbatim}
def machineEpsilon(func=float):
    machine_epsilon = func(1)
    while func(1)+func(machine_epsilon) != func(1):
        machine_epsilon_last = machine_epsilon
        machine_epsilon = func(machine_epsilon) / func(2)
    return machine_epsilon_last
\end{verbatim}
\normalsize
Understand what the function does,  and determine $\epsilon$ for \verb!int!,  \verb!float!, and \verb!complex!. How is \verb!float! in Python represented, single or double precision? 



\item To convert a float pointing number  between the machine representation in hexadecimals and the decimal  representation using Python, we will use the \verb!struct! module in Python. In this exercise, we will  use two functions in the module, \verb!pack! and \verb!unpack!,  to perform the task (\url{http://docs.python.org/2/library/struct.html}).
The following code perform conversion of a hexadecimal representation of bit pattern \verb!BF4F9680! to an unsigned integer,  
\begin{verbatim}
import struct
output=struct.unpack("I", struct.pack("I",0xBF4F9680))[0]
print output
\end{verbatim}
 What if you change the format string \verb!"I"!  in \verb!unpack!  to \verb!"i"!? How about to \verb!"f"!?

Use the program to 
\begin{enumerate} 


\item Determine the decimal numbers that have the following machine representations: a. [3F27E520]$_{16}$,\;	b. [3BCDCA00]$_{16}$,\;	c. [BF4F9680]$_{16}$.
\item Determine the machine representation in hexadecimals in IEEE single precision for the following decimal numbers:
a. $2^{−30}$,\;	b. 64.015625,\;	c. $−8\times 2^{−24}$. Use the bulid-in functions \verb!hex()! to convert an unsigned integer to hexadecimals, and \verb!bin()! to binaries.
\end{enumerate}
\end{enumerate}
\end{document}	
