\documentclass[12pt]{article}
\usepackage{times}
\usepackage{amsmath}
\usepackage{a4}
\usepackage{graphicx}
\setlength{\oddsidemargin}{-0.2in}
%\setlength{\evensidemargin}{-0.5in}
\setlength{\textwidth}{6.5in}
\begin{document}
\begin{center}
\Large
\textbf{Numerical Analysis and Programming}\\
\large
Lab Worksheet \#6
\end{center}
\begin{enumerate}
\item \textit{List Comprehension}
List comprehensions provide a concise way to create lists. Each list comprehension consists of an expression followed by a  {\tt for} clause, then zero or more {\tt for} or {\tt if} clauses. The result will be a list resulting from evaluating the expression in the context of the  {\tt for} and {\tt if}  clauses which follow it. If the expression would evaluate to a tuple, it must be parenthesized. Here are some examples:
\small
\begin{verbatim}
>>> vec = [2, 4, 6]
>>> [3*x for x in vec]
[6, 12, 18]
>>> [3*x for x in vec if x > 3]
[12, 18]
>>> [3*x for x in vec if x < 2]
[]
>>> [[x,x**2] for x in vec]
[[2, 4], [4, 16], [6, 36]]
\end{verbatim}
\normalsize
Using list comprehension to generate the transpose matrix and trace of the matrix,
\small
\begin{verbatim}
>>> M=[[1,2,3],
   [4,5,6],
   [7,8,9]]
\end{verbatim}
\item \textit{Data Analysis} The data file \verb!Data.in! contains the exam scores of a class, in the comma-separated-value (csv) format.
Each row corresponds to the scores of a student, and each column is  the score that the student got for each problem. Write a short program to perform the following tasks:
\item Find the maximum, minimum, average, median, and standard deviation of the exam scores, using \verb!numpy! supplied functions.
\item Use the function \verb!hist! in \textit{matplotlib}  to plot the histogram of the score distribution with 10-point bin intervals.

\end{enumerate}
\end{document}
