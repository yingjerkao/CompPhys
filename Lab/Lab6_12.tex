\documentclass[12pt]{article}
\usepackage{times}
\usepackage{amsmath}
\usepackage{a4}
\usepackage{graphicx}
\setlength{\oddsidemargin}{-0.2in}
%\setlength{\evensidemargin}{-0.5in}
\setlength{\textwidth}{6.5in}
\begin{document}
\begin{center}
\Large
\textbf{Numerical Analysis and Programming}\\
\large
Lab Worksheet \#6
\end{center}
\begin{enumerate}
\item This exercise will be useful for problem 1 in the assignment. The following are functions that take a string argument and
return the first, last, and middle letters:


\begin{verbatim}
def first(word):
    return word[0]

def last(word):
    return word[-1]

def middle(word):
    return word[1:-1]
\end{verbatim}


 Type these functions into a file named {\tt palindrome.py}
and test them out.  What happens if you call {\tt middle} with
a string with two letters?  One letter?  What about the empty
string, which is written \verb"''" and contains no letters?
\item Use string methods to write a \verb!has_palindrome! function, which find the part of a string which is a palindrome. For example, 'afternoon' is not a palindrome, but 'noon' is. Test your function with 'afternoon', and  'class'.   

\end{enumerate}
\end{document}	
