\documentclass[12pt]{article}
\usepackage{times}
\usepackage{amsmath}
\usepackage{a4}
\usepackage{graphicx}
\setlength{\oddsidemargin}{-0.2in}
\pagestyle{empty}
%\setlength{\evensidemargin}{-0.5in}
\setlength{\textwidth}{6.5in}
\begin{document}
\begin{center}
\Large
\textbf{Numerical Analysis and Programming}\\
\large
Lab Worksheet \#5
\end{center}
\begin{enumerate}
\item The following function approximately determines the \textit{machine epsilon} ($\epsilon$) for a given type (defaults to \verb!float!), using the definition that $\epsilon$ is the smallest positive number such that $1+\epsilon\ne 1$. 
\small
\begin{verbatim}
def machineEpsilon(func=float):
    machine_epsilon = func(1)
    while func(1)+func(machine_epsilon) != func(1):
        machine_epsilon_last = machine_epsilon
        machine_epsilon = func(machine_epsilon) / func(2)
    return machine_epsilon_last
\end{verbatim}
\normalsize
Understand what the function does,  and determine $\epsilon$ for \verb!int!,  \verb!float!, and \verb!complex!. How is \verb!float! in Python represented, single or double precision? 


\item Determine the machine representation in hexidecimals in IEEE single precision for the following decimal numbers:
a. $2^{−30}$,\;	b. 64.015625,\;	c. $−8\times 2^{−24}$.
%You can check your results using \verb!float.hex()! function in Python.

\item Determine the decimal numbers that have the following machine representations: a. [3F27E520]$_{16}$,\;	b. [3BCDCA00]$_{16}$,\;	c. [BF4F9680]$_{16}$.
\item In the homework, you are required to write programs to perform the conversions above. Explore the \verb!struct! module in Python. Learn to use two functions in the module: \verb!pack! and \verb!unpack!. 
Try the following code
\begin{verbatim}
import struct
ss=struct.pack("I",0xBF4F9680)
print "ss", ss
output, =struct.unpack("I", ss)
print output
\end{verbatim}
What does the code do? What if you change the format string \verb!"I"! to \verb!"i"!?
%You can check your results using \verb!float.fromhex()! function in Python.  Notice a prefix \verb!0x! needs to be added in front of a hexadecimal number in Python.
\end{enumerate}
\end{document}	
