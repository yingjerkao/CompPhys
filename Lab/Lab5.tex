\documentclass[12pt]{article}
\usepackage{times}
\usepackage{amsmath}
\usepackage{a4}
\usepackage{graphicx}
%\setlength{\oddsidemargin}{-0.2in}
%\setlength{\evensidemargin}{-0.5in}
%\setlength{\textwidth}{6in}
\begin{document}
\begin{center}
\Large
\textbf{Numerical Analysis and Programming}\\
\large
Lab Worksheet \#5
\end{center}
\begin{enumerate}
\item \textit{Palindrome} A palindrome is a word that is spelled the same backward and
forward, like ``noon'' and ``redivider''.  Recursively, a word
is a palindrome if the first and last letters are the same
and the middle is a palindrome. 

\begin{enumerate} 
\item Write a function defined as
\begin{verbatim}
def is_palindrome(s):
\end{verbatim}
 that takes
a string argument \verb!s! and returns {\tt True} if it is a palindrome
and {\tt False} otherwise using string methods.  
\item Use this function to write a function \verb"has_palindrome" defined as 
\begin{verbatim}
def has_palindrome(s, start, length):
\end{verbatim}
which determines if the  substring of string \verb!s! starting from \verb!start! with length \verb!length! is palindrome.  
\item Use this function to solve the following puzzler:
\begin{quote}
``I was driving on the highway the other day and I happened to
notice my odometer. Like most odometers, it shows six digits,
in whole miles only. So, if my car had 300,000
miles, for example, I'd see 3-0-0-0-0-0.

``Now, what I saw that day was very interesting. I noticed that the
last 4 digits were palindromic; that is, they read the same forward as
backward. For example, 5-4-4-5 is a palindrome, so my odometer
could have read 3-1-5-4-4-5.

``One mile later, the last 5 numbers were palindromic. For example, it
could have read 3-6-5-4-5-6.  One mile after that, the middle 4 out of
6 numbers were palindromic.  And you ready for this? One mile later,
all 6 were palindromic!

``The question is, what was on the odometer when I first looked?''
\end{quote}
Write a function \verb!puzzle_solver()! which  tests all six-digit numbers and return the possible numbers that satisfy these requirements.


\end{enumerate}
 \item  \textit{Using Lists as Stacks}
The list methods make it very easy to use a list as a stack, where the last element added is the first element retrieved (''last-in, first-out''). To add an item to the top of the stack, use \texttt{append()}. To retrieve an item from the top of the stack, use \texttt{pop()} without an explicit index. It is easy to implement non-recursive version of recursive function using stacks. Implement \texttt{factorial(n)} for non-negative integer $n$ using a stack. You may want to go back to study the recursive version of \texttt{factorial(n)}.

\end{enumerate}
\end{document}	
