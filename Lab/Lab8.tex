\documentclass[12pt]{article}
\usepackage{times}
\usepackage{amsmath}
\usepackage{a4}
\usepackage{graphicx}
\usepackage{hyperref}
%\setlength{\oddsidemargin}{-0.2in}
%\setlength{\evensidemargin}{-0.5in}
%\setlength{\textwidth}{6.5in}
\begin{document}
\begin{center}
\Large
\textbf{Numerical Analysis and Programming}\\
\large
Lab Worksheet \#8\\
\end{center}
We will use the \verb|Gui| module provided in Think Python to experiment how to program a graphical user interface (GUI).
It is a wrapper for Python's  GUI module \verb!Tkinter!, which provides all sorts of widgets that you can use for GUI programming. 
Documentation of \verb!Gui! is available at \url{http://www.greenteapress.com/thinkpython/swampy/Gui.html}

Make sure that  \verb|Gui.py| and \verb!Lab8.py! is in the same directory. \verb!Lab8.py! provides a typical example of a GUI application. In this case, the program takes an integer $n$ as input and compute the factorial of $n$. We define a class called \verb!FactorialGUI! with several methods. Notice that the main event loop is implemented inside  \verb!__init__()!.  There are three widgets: one ``Compute" button, one entry for input and one ``Quit" button.  

 
\begin{enumerate}
\item Test the application with several integers. You can either press Enter or press the Compute button to perform computation. What happens if your input is not an integer?
\item  There are two callback functions: \verb!compute! for the Compute button widget and \verb!compute2! which is bound to the keyboard \textit{event} of pressing Return/Enter. \verb!compute2! takes an extra parameter \verb!event!, which is an object instance with a number of attributes describing the event. \verb!event.x!, and \verb!event.y! give the $x$ and $y$ coordinate of the mouse position when the event (pressing Return) occurs. Remove \verb!###! in the first line of  \verb!compute2!, and try several inputs with different cursor positions. Observe the output at your python shell.  What happens if you use the Compute button instead?    
\item Try to resize the window by click-drag the shaded triangle at the bottom-right corner of the window. What happens to the widgets? If you this line of code  \verb!self.g.gr(10,[0,0,0],[0])! to \verb!self.g.gr(10,[0,0,1],[0])!? Experiment with several inputs for \verb!gr!.
Use \verb!help(Gui.gr)! to find the meaning of the parameters.
\item Now the output text is inside the entry box, which is a bit annoying. Use \verb!Gui.te()! to create a text entry widget to print the output in the format of \verb|6!=24|. Position the text entry above  all the buttons we created. What happens if the screen becomes full? You might want to resize the window or use a scrollable text entry. 
\end{enumerate}

\end{document}	
