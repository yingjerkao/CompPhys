\documentclass[12pt]{article}
\usepackage{times}
\usepackage{amsmath}
\usepackage{a4}
\usepackage{graphicx}
\setlength{\oddsidemargin}{-0.2in}
%\setlength{\evensidemargin}{-0.5in}
\setlength{\textwidth}{6.5in}
\begin{document}
\begin{center}
\Large
\textbf{Numerical Analysis and Programming}\\
\large
Lab Worksheet \#12
\end{center}
 We will simulate a spontaneous decay process of nuclei in this exercise. The decay rate of the radioactive nuclei is given by
\[
\frac{dN}{dt}=-\frac{1}{\tau} N
\]
where $N$ is the number of undecayed nuclei at time $t$, and $\tau$ is the life-time.
Rewrite the equation in the discrete time, and set $\Delta t=1$, we have
\[
\Delta N = -\frac{1}{\tau} N =-\lambda N
\]
The probability of a nucleus to decay per generation is then $\lambda$ with $0\le \lambda\le 1$.
\begin{enumerate}
\item Assume at the beginning we have $N_0=10000$ Co$^{60}$ nuclei with a half-life $t_{1/2}=  5.27$ years. Notice $t_{1/2}=\tau \ln(2)$. Write a program to simulate
the decay process with the updating scheme:
\begin{enumerate}
\item Scan through each nucleus. Examine whether the nucleus has decayed. If not, generate a random number $r$ between 0 and 1.
\item If $r < \lambda $ then mark the nucleus as decayed, otherwise the nucleus remains undecayed.
\item Repeat the process until no nucleus is found undecayed.
\end{enumerate}
Assume $\Delta t $= 1 month and monitor the number of the undecayed  nuclei. Estimate the half-life from your data.
\end{enumerate}
\end{document}
