\documentclass[12pt]{article}
\usepackage{times}
\usepackage{amsmath}
\usepackage{a4}
\usepackage{graphicx}
%\setlength{\oddsidemargin}{-0.5in}
%\setlength{\evensidemargin}{-0.5in}
%\setlength{\textwidth}{in}
\usepackage{hyperref}
\begin{document}
\begin{center}
\Large
\textbf{Numerical Analysis and Programming}\\
\large
Lab Worksheet \#1
\end{center}
\begin{enumerate}
\item Familiarize yourself with IDLE. You can consult the user manual at \url{http://docs.python.org/2/library/idle.html}.
\item Start the Python interpreter and type \texttt{help()} to start the online
help utility.  For example,  you can type \verb"help('keywords')" to get all the Python keywords.
\item Start the Python interpreter and use it as a calculator.
Python's syntax for math operations is almost the same as
standard mathematical notation.  For example, the symbols
\texttt{+}, \texttt{-} and \texttt{/} denote addition, subtraction
and division, as you would expect.  The symbol for
multiplication is \texttt{*}.
\begin{enumerate}
\item Find the definition of operators  \texttt{//} and \texttt{\%} from the python documentation.
\item Find the values of \texttt{5/3} and \texttt{5.0/3}.
\item Find the values of \texttt{5//3}  and \texttt{5.0//3}. 
\item Find the values of \texttt{5\%3} and \texttt{5.0\%3}. 
\item Find the values of \texttt{5.0/-3}, \texttt{5.0//-3} and \texttt{5.0\%-3}.
\item Use the built-in function \texttt{divmod()} to find the quotient and remainder of \texttt{5.0/-3}. You can use \texttt{help} to find the definition of \texttt{divmod()}.  
\end{enumerate}
\item You need to import the \texttt{math} library to perform the following tasks.
\begin{verbatim}
>>> import math
\end{verbatim} 
\begin{enumerate}
\item Explore the math library by using the command {\tt dir}. Print out the value of $\sin (0.3\pi)$.
\item Evaluate the expressions \texttt{math.floor(5.0/3)} and \texttt{math.floor(5.0/-3)}.
\item Show that \texttt{a//b=math.floor(a/b)} and \texttt{a\%b=a-math.floor(a/b)*b}.
\end{enumerate}
\end{enumerate}
\end{document}
