\documentclass[12pt]{article}
\usepackage{times}
\usepackage{amsmath}
\usepackage{a4}
\usepackage{graphicx}
\setlength{\oddsidemargin}{-0.2in}
%\setlength{\evensidemargin}{-0.5in}
\setlength{\textwidth}{6.5in}
\begin{document}
\begin{center}
\Large
\textbf{Numerical Analysis and Programming}\\
\large
Lab Worksheet \#4
\end{center}
\begin{enumerate}

\item  In this example, we will  see how to exploit the properties of a mathematical function  to obtain a more rapidly convergent series. To calculate an approximate value of $\ln 2$, we may use the Talyor series expansion,
\[
\ln (1+x) =\sum_{n=1} (-1)^{n+1}\frac{1}{n}x^n.
\]
Write a program to sum over the first $n$ terms of the series. What is the value of this sum when $n=10$ ? Is it a good approximation of  the value obtained by \verb'log(2)' function in the math module? 

One can try another series
 \[
\ln \frac{1+x}{1-x} =\sum_{n=1} \frac{2}{2n-1}x^{2n-1},
\]
with $x=1/3$. How many terms do you need to keep to obtain the value which agrees with the value from \verb'log(2)' to six decimal places? Why is this better ? 

\end{enumerate}
\end{document}	
